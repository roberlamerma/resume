%%%%%%%%%%%%%%%%%
% This is an sample CV template created using altacv.cls
% It is based on (v1.1.3, 30 April 2017) written by LianTze Lim (liantze@gmail.com).
% It was been heavily modified since its initial cloning
% 
%% It may be distributed and/or modified under the
%% conditions of the LaTeX Project Public License, either version 1.3
%% of this license or (at your option) any later version.
%% The latest version of this license is in
%%    http://www.latex-project.org/lppl.txt
%% and version 1.3 or later is part of all distributions of LaTeX
%% version 2003/12/01 or later.
%%%%%%%%%%%%%%%%

%% Defining languages. Got this here: https://ptmartins.info/2021/05/15/cv.html
\newif\ifen
\newif\ifpt
\entrue % English enabled (either this line or the one below should be enabled, not both!)
%\pttrue % Portuguese enabled

%% If you need to pass whatever options to xcolor
\PassOptionsToPackage{dvipsnames}{xcolor}

%% If you are using \orcid or academicons
%% icons, make sure you have the academicons 
%% option here, and compile with XeLaTeX
%% or LuaLaTeX.
% \documentclass[10pt,a4paper,academicons]{altacv}

%% Use the "normalphoto" option if you want a normal photo instead of cropped to a circle
% \documentclass[10pt,a4paper,normalphoto]{altacv}

\documentclass[10pt,a4paper]{altacv}
%% AltaCV uses the fontawesome and academicon fonts
%% and packages. 
%% See texdoc.net/pkg/fontawecome and http://texdoc.net/pkg/academicons for full list of symbols.
%% 
%% Compile with LuaLaTeX for best results. If you
%% want to use XeLaTeX, you may need to install
%% Academicons.ttf in your operating system's font 
%% folder.

% For Languages
\usepackage[english,brazilian]{babel}
\usepackage{csquotes} % recommended when using babel

\usepackage{hyphenat}

%language markup commands
\newcommand{\langen}[1]{%
	\ifen\selectlanguage{english}#1\fi}
\newcommand{\langpt}[1]{%
	\ifpt\selectlanguage{portuges}#1\fi}

%% For SVG inline images
\usepackage{svg}

% Change the page layout if you need to
\geometry{left=1cm,right=9cm,marginparwidth=6.8cm,marginparsep=1.2cm,top=1.25cm,bottom=1.25cm,footskip=2\baselineskip}

% Change the font if you want to.

% If using pdflatex:
\usepackage[T1]{fontenc}
\usepackage[utf8]{inputenc}
\usepackage[default]{lato}

% For justifying
\usepackage{ragged2e}

% If using xelatex or lualatex:
% \setmainfont{Lato}

% Change the colours
\definecolor{Emphasis}{HTML}{4a4d9e}
\definecolor{Accent}{HTML}{0f0f45}
\definecolor{Body}{HTML}{555555}
\definecolor{Heading}{HTML}{181741}
\colorlet{heading}{Heading}
\colorlet{accent}{Accent}
\colorlet{emphasis}{Emphasis}
\colorlet{body}{Body}

% Change the bullets for itemize and rating marker
% for \cvskill if you want to
\renewcommand{\itemmarker}{{\small\textbullet}}
\renewcommand{\ratingmarker}{\faCircle}

%% sample.bib contains your publications
\addbibresource{sample.bib}

\usepackage[colorlinks]{hyperref}
\hypersetup{
	colorlinks=true,    
	urlcolor=blue,
}

\begin{document}

\name{Roberto Arias Ruiz}
\tagline{Software Product Manager}
\photo{2.5cm}{Rober_Gray_Round}
\personalinfo{%
  % Not all of these are required!
  % You can add your own with \printinfo{symbol}{detail}
  \mailaddress{roberlamerma@gmail.com}
  % \phone{+41 774967886 / +34 678714822}
  \phone{+55 47 99942 7456}
     
%  \mailaddress{Sonnhaldenstrasse 25}
%  \location{Sonnhaldenstrasse 25, 6331 Hünenberg, Switzerland}
%  \linkedin{linkedin.com/in/roberlamerma}
  
 \mbox{\linkedin{\href{https://www.linkedin.com/in/roberlamerma/}{roberlamerma}} \printinfo{\includesvg[width=0.09in]{readcv.svg}}{{\href{https://read.cv/roberlamerma}{Read.cv}}}}  

  % \github{github.com/roberlamerma}
  %% You MUST add the academicons option to \documentclass, then compile with LuaLaTeX or XeLaTeX, if you want to use \orcid or other academicons commands.
%   \orcid{orcid.org/0000-0000-0000-0000}

}

%% Make the header extend all the way to the right, if you want. 
\begin{fullwidth}
\makecvheader
\end{fullwidth}

%% Provide the file name containing the sidebar contents as an optional parameter to \cvsection.
%% You can always just use \marginpar{...} if you do
%% not need to align the top of the contents to any
%% \cvsection title in the "main" bar.
%\cvsection[page1sidebar]{Profile}
\cvsection[page1sidebar]{\langen{Profile}\langpt{Perfil}}

\begin{itemize} 
	\langen{\item Over 20 years of comprehensive experience in all stages of the Software Development Life Cycle: from conception to delivery.} 
	\langpt{\item Mais de 20 anos de experiência em todas as etapas do Ciclo de Vida do Desenvolvimento de Software: da concepção à entrega.}
	
	\langen{\item Ability to bridge the gap between technical teams and business stakeholders, ensuring that projects meet both strategic goals and user needs.} 
	\langpt{\item Capacidade de conectar equipes técnicas e stakeholders de negócios, garantindo que os projetos atendam tanto aos objetivos estratégicos quanto às necessidades dos usuários.}
	
	\langen{\item Expert in Agile methodologies (SAFe, Scrum) with a proven track record of successful implementations.} 
	\langpt{\item Especialista em metodologias ágeis (SAFe, Scrum) com um histórico comprovado de implementações bem-sucedidas.}
	
	\langen{\item Exceptional communication and presentation skills, with the ability to engage both technical and non-technical audiences.} 
	\langpt{\item Habilidades excepcionais de comunicação e apresentação, com a capacidade de engajar públicos técnicos e não técnicos.}
	
	\langen{\item Skilled in managing high-pressure situations with data and fact-driven decision-making.} 
	\langpt{\item Habilidade em gerenciar situações de alta pressão com tomada de decisão baseada em dados e fatos.}
\end{itemize}

\medskip

\cvsection{\langen{Experience}\langpt{Experiência}}

%\cvevent{Product Owner}{Roche Diagnostics}{May 2011 -- Ongoing}{Barcelona, Spain}
%\cvevent{Roche Diagnostics \href{https://www.roche.com/about/business/diagnostics.htm}{\small <http://www.roche.com>}\par}{Product Owner}{May 2015 -- Ongoing}{Barcelona, Spain}
\cvevent{Roche Diagnostics}{Product Manager}{Nov 2022 -- Jan 2025}{Rotkreuz, Switzerland}
\begin{itemize}
	\langen{\item SW Product Manager for \textbf{Roche's first DNA-Sequencing instrument}.} 
	\langpt{\item SW Product Manager para \textbf{o primeiro instrumento de sequenciamento de DNA da Roche}.}
	
	\langen{\item Developed project plans, including the product roadmap, work breakdown structure, and overall software vision.} 
	\langpt{\item Desenvolveu planos de projeto, incluindo o roadmap do produto, a estrutura de decomposição do trabalho e a visão geral do software.}
	
	\langen{\item Worked together with various stakeholders in order to define the right features for each of our milestones.} 
	\langpt{\item Trabalhou em conjunto com diversos stakeholders para definir as funcionalidades adequadas para cada um de nossos marcos.}
	
	\langen{\item Recruited and formed the SW development teams.} 
	\langpt{\item Recrutou e formou as equipes de desenvolvimento de software.}
\end{itemize}


\cvevent{}{Product Owner}{\langen{May}\langpt{Maio} 2015 -- Nov 2022}{Rotkreuz, Switzerland}
\begin{itemize}
	\langen{\item Led product development for the {\href{https://www.cobasliat.com/}{\textsuperscript{\textregistered}Cobas \textsuperscript{\textregistered}Liat}} point-of-care device, which was \textbf{pivotal during the COVID-19 pandemic}.}  
	\langpt{\item Liderou o desenvolvimento do produto para o dispositivo point-of-care {\href{https://www.cobasliat.com/}{\textsuperscript{\textregistered}Cobas \textsuperscript{\textregistered}Liat}}, que foi \textbf{fundamental durante a pandemia de COVID-19}.}  
	
	\langen{\item Defined and prioritized user stories for the SW development teams and coordinated acceptance testing with QA teams in a distributed setup (Spain and Switzerland).}  
	\langpt{\item Definiu e priorizou user stories para as equipes de desenvolvimento de software e coordenou os testes de aceitação com as equipes de QA em uma configuração distribuída (Espanha e Suíça).}  
	
	\langen{\item Created detailed requirements and user interface sketches to guide development.}  
	\langpt{\item Criou requisitos detalhados e esboços de interface do usuário para orientar o desenvolvimento.}   
\end{itemize}


\cvevent{}{Scrum \& Build Master / \langen{Senior Software Developer}\langpt{Desenvolvedor de Software Sênior}}{\langen{May}\langpt{Maio} 2011 -- \langen{May}\langpt{Maio} 2015}{Barcelona, \langen{Spain}\langpt{Espanha}}
\begin{itemize}
	\langen{\item Led a local Scrum team and coordinated with international teams in Spain, India, and Switzerland to deliver a high-throughput PCR analyzer.}  
	\langpt{\item Liderou uma equipe local de Scrum e coordenou com equipes internacionais na Espanha, Índia e Suíça para entregar um analisador de PCR de alto rendimento.}  
	
	\langen{\item Developed a \textbf{comprehensive reporting and KPI analysis suite} using Microsoft Reporting/Integration Services, TFS, SQL, OLAP cubes, and PowerBI.}  
	\langpt{\item Desenvolveu um \textbf{conjunto abrangente de relatórios e análise de KPIs} usando Microsoft Reporting/Integration Services, TFS, SQL, cubos OLAP e PowerBI.}  
\end{itemize}

\divider


\cvevent{Wolters Kluwer}{\langen{Software Architect}\langpt{Arquiteto de Software}}{\langen{Apr}\langpt{Abril} 2010 -- \langen{May}\langpt{Maio} 2011}{Barcelona, \langen{Spain}\langpt{Espanha}}
\begin{itemize}
	\langen{\item Designed and developed a \textbf{C{\#}-based SW framework} for internal tax and accounting applications, enhancing operational efficiency.}  
	\langpt{\item Design e desenvolveu um \textbf{framework de software baseado em C{\#}} para aplicações internas de contabilidade e tributação, aumentando a eficiência operacional.}  
\end{itemize}

\vfil \break

% \divider
\cvsection[page2sidebar]{\langen{Experience}\langpt{Experiência}}


\cvevent{Amper}{\langen{Senior Software Developer}\langpt{Desenvolvedor de Software Sênior}}{\langen{Mar}\langpt{Março} 2007 -- Feb 2010}{Barcelona, \langen{Spain}\langpt{Espanha}}
\begin{itemize}
	\langen{\item Conducted \textbf{research and development} on embedded systems and comm protocols, driving innovation in the company’s product offerings.}  
	\langpt{\item Conduziu \textbf{pesquisa e desenvolvimento} em sistemas embarcados e protocolos de comunicação, impulsionando a inovação nos produtos da empresa.}  
	
	\langen{\item Designed and developed .NET services and applications.}  
	\langpt{\item Design e desenvolveu serviços e aplicações .NET.}  
\end{itemize}

\divider


\cvevent{Grupo 3C}{\langen{Software Developer}\langpt{Desenvolvedor de Software}}{\langen{Dec}\langpt{Dez} 2005 -- Jan 2007}{Barcelona, \langen{Spain}\langpt{Espanha}}
\begin{itemize}
	\langen{\item Developed \textbf{web applications} using PHP, Perl, MySQL, HTML, CSS, and JavaScript, improving client service delivery.}  
	\langpt{\item Desenvolveu \textbf{aplicações web} usando PHP, Perl, MySQL, HTML, CSS e JavaScript, melhorando a prestação de serviços aos clientes.}  
\end{itemize}

\divider


\cvevent{iChameleon Group INC}{\langen{Full-stack Software Developer}\langpt{Desenvolvedor de Software "Full-stack"}}{\langen{Aug}\langpt{Ago} 2004 -- \langen{Oct}\langpt{Out} 2005}{(Freelancer - \langen{several locations}\langpt{paises diferentes})}
\begin{itemize}
	\langen{\item Designed and implemented SW for an \textbf{online music shop} as a full-stack developer, utilizing Perl, FastCGI, PostgreSQL, and Objective-C.}  
	\langpt{\item Design e desenvolveu software para loja de música online usando Perl, FastCGI, PostgreSQL e Objective-C.}  
	% Projetou e implementou software para uma \textbf{loja de música online} como desenvolvedor full-stack, utilizando Perl, FastCGI, PostgreSQL e Objective-C.
\end{itemize}

\divider


\cvevent{Gaiax Co, LTD}{\langen{Software Developer Internship}\langpt{Estágio de Desenvolvedor de Software}}{\langen{Aug}\langpt{Ago} 2003 -- Jul 2004}{\langen{Tokyo, Japan}\langpt{Tóquio, Japão}}
\begin{itemize}
	\langen{\item Developed a Web game using Perl (mod\_perl and FastCGI). This was also \textbf{my university's thesis research project} on high loading \& demand technologies.}  
	\langpt{\item Desenvolveu um jogo web com Perl (mod\_perl e FastCGI), parte da sua tese sobre tecnologias de alta carga e demanda.}  
\end{itemize}

\divider


\cvevent{\langen{Computing Laboratory}\langpt{Laboratório de informática}, \langen{Simón Bolívar University}\langpt{Universidade Simón Bolívar}}{IT \& Systems Administrator}{\langen{Oct}\langpt{Out} 1999 -- \langen{Oct}\langpt{Out} 2002}{Caracas, Venezuela}
\begin{itemize}
	\langen{\item Managed and maintained parts of the \textbf{university’s LAN and lab servers} across Linux, Solaris, and Windows platforms, ensuring operational continuity and reliability.}  
	\langpt{\item Gerenciou e manteve partes da \textbf{LAN e dos servidores de laboratório} da universidade em plataformas Linux, Solaris e Windows, garantindo continuidade operacional e confiabilidade.}  
	
	\langen{\item Coordinated the admission of new members (exams, training, mentoring).}  
	\langpt{\item Coordenou a admissão de novos membros (exames, treinamento, mentoria).}  
\end{itemize}

\medskip
\medskip
%\medskip
%\medskip

% Adapted from @Jake's answer from http://tex.stackexchange.com/a/82729/226
% \wheelchart{outer radius}{inner radius}{
% comma-separated list of value/text width/color/detail}
\langen{\cvsection{Product management skills}}  
\langpt{\cvsection{\large Habilidades em gestão de produtos}}

\langen{\wheelchart{1.3cm}{0.4cm}{
		3/8em/Emphasis!90/Stakeholder management,  
		2/8em/Emphasis!45/KPI analysis,  
		3/8em/Emphasis/Communication,  
		3/4em/Emphasis!75/Usability \& UX,  
		2/5em/Emphasis!30/\mbox{Requirements engineering},  
		2/10em/Emphasis!15/Data-driven decision making,  
		3/8em/Emphasis!60/Story mapping \& Roadmaps  
}}  
\langpt{\wheelchart{1.3cm}{0.4cm}{
		3/8em/Emphasis!90/Gerenciamento de stakeholders,  
		2/8em/Emphasis!45/Análise de KPIs,  
		3/8em/Emphasis/Comunicação,  
		3/4em/Emphasis!75/Usabilidade \& UX,  
		2/5em/Emphasis!30/\mbox{Engenharia de requisitos},  
		2/10em/Emphasis!15/Tomada de decisão baseada em dados,  
		3/8em/Emphasis!60/Story mapping \& Roadmaps  
}}  


%\langen{\cvsection{Agile competencies}}  
%\langpt{\cvsection{\large Competências "Agile"}}
%
%\langen{\wheelchart{1.3cm}{0.4cm}{
%		3/8em/Emphasis!90/Lean-Agile Leadership,  
%		2/8em/Emphasis!45/Expertise forming and managing Agile teams,  
%		3/8em/Emphasis/DevOps and Release on Demand,  
%		3/4em/Emphasis!75/\mbox{Business Solutions},  
%		2/5em/Emphasis!30/\mbox{Lean Systems Engineering},  
%		2/10em/Emphasis!15/People leader  
%}}  
%\langpt{\wheelchart{1.3cm}{0.4cm}{
%		3/8em/Emphasis!90/Liderança Lean-Agile,  
%		2/8em/Emphasis!45/Expertise na \mbox{formação e gestão} \mbox{de equipes "Agile"},  
%		3/8em/Emphasis/DevOps e "Release on Demand",  
%		3/4em/Emphasis!75/\mbox{Soluções de negócios},  
%		2/5em/Emphasis!30/\mbox{Engenharia de} \mbox{"Sistemas Lean"},  
%		2/10em/Emphasis!15/Liderança de pessoas  
%}}  

\langen{\cvsection{Technical Skills}}  
\langpt{\cvsection{\large Habilidades técnicas}}

\langen{\wheelchart{1.3cm}{0.4cm}{
	2/8em/Emphasis!90/Software Architecture,  
	3/8em/Emphasis!45/Linux expert,  
	2/8em/Emphasis!15/Artificial Intelligence, 
	3/8em/Emphasis/\mbox{C\# \& C++},  
	2/4em/Emphasis!75/\mbox{Cloud computing},  
	2/5em/Emphasis!30/\mbox{Containerization} \mbox{and Virtualization},  
	2/10em/Emphasis!15/Scripting: \mbox{Bash \& Python},  
	2/8em/Emphasis!60/Embedded \mbox{Linux (ARM)}
 }}
\langpt{\wheelchart{1.3cm}{0.4cm}{
		2/8em/Emphasis!90/Arquitetura de software,  
		3/8em/Emphasis!45/Especialista em Linux,  
		2/8em/Emphasis!15/Inteligência artificial, 
		3/8em/Emphasis/\mbox{C\# \& C++},  
		2/4em/Emphasis!75/\mbox{Computação em nuvem},  
		2/5em/Emphasis!30/\mbox{Containerization} \mbox{e Virtualization},  
		2/10em/Emphasis!15/\mbox{Bash \& Python},  
		2/8em/Emphasis!60/Embedded \mbox{Linux (ARM)}
}}
%\cvsection{Programming skills}
% \wheelchart{1.5cm}{0.5cm}{%
% 	7/2em/Emphasis!80/C\#, 
% 	5/2em/Emphasis!50/Perl,
% 	3/2em/Emphasis!40/C++,
% 	3/2em/Emphasis/Bash,
% 	2/2em/Emphasis!60/Phyton,
% 	4/2em/Emphasis!20/{HTML/CSS/JS}
% }

% \cvsection{OS skills}

% \wheelchart{1.5cm}{0.5cm}{%
% 	10/2em/Emphasis!30/Linux, 
% 	3/2em/Emphasis!20/Mac OS,
% 	8/2em/Emphasis!60/Windows,
% 	5/2em/Emphasis/Embedded Linux
% }

%\colorlet{heading}{Heading}
%\colorlet{accent}{Accent}
%\colorlet{emphasis}{Emphasis}
%\colorlet{body}{Body}

%\clearpage
%\cvsection[page2sidebar]{Publications}

%\nocite{*}

%\printbibliography[heading=pubtype,title={\printinfo{\faBook}{Books}},type=book]

%\divider

%\printbibliography[heading=pubtype,title={\printinfo{\faFileTextO}{Journal Articles}},type=article]

%\divider

%\printbibliography[heading=pubtype,title={\printinfo{\faGroup}{Conference Proceedings}},type=inproceedings]

%% If the NEXT page doesn't start with a \cvsection but you'd
%% still like to add a sidebar, then use this command on THIS
%% page to add it. The optional argument lets you pull up the 
%% sidebar a bit so that it looks aligned with the top of the
%% main column.
% \addnextpagesidebar[-1ex]{page3sidebar}

\end{document}
