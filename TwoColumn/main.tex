%%%%%%%%%%%%%%%%%
% This is an sample CV template created using altacv.cls
% (v1.1.3, 30 April 2017) written by LianTze Lim (liantze@gmail.com). Now compiles with pdfLaTeX, XeLaTeX and LuaLaTeX.
% 
%% It may be distributed and/or modified under the
%% conditions of the LaTeX Project Public License, either version 1.3
%% of this license or (at your option) any later version.
%% The latest version of this license is in
%%    http://www.latex-project.org/lppl.txt
%% and version 1.3 or later is part of all distributions of LaTeX
%% version 2003/12/01 or later.
%%%%%%%%%%%%%%%%

%% If you need to pass whatever options to xcolor
\PassOptionsToPackage{dvipsnames}{xcolor}

%% If you are using \orcid or academicons
%% icons, make sure you have the academicons 
%% option here, and compile with XeLaTeX
%% or LuaLaTeX.
% \documentclass[10pt,a4paper,academicons]{altacv}

%% Use the "normalphoto" option if you want a normal photo instead of cropped to a circle
% \documentclass[10pt,a4paper,normalphoto]{altacv}

\documentclass[10pt,a4paper]{altacv}
%% AltaCV uses the fontawesome and academicon fonts
%% and packages. 
%% See texdoc.net/pkg/fontawecome and http://texdoc.net/pkg/academicons for full list of symbols.
%% 
%% Compile with LuaLaTeX for best results. If you
%% want to use XeLaTeX, you may need to install
%% Academicons.ttf in your operating system's font 
%% folder.

\usepackage{svg}

% Change the page layout if you need to
\geometry{left=1cm,right=9cm,marginparwidth=6.8cm,marginparsep=1.2cm,top=1.25cm,bottom=1.25cm,footskip=2\baselineskip}

% Change the font if you want to.

% If using pdflatex:
\usepackage[T1]{fontenc}
\usepackage[utf8]{inputenc}
\usepackage[default]{lato}

% For justifying
\usepackage{ragged2e}

% If using xelatex or lualatex:
% \setmainfont{Lato}

% Change the colours if you want to
\definecolor{Emphasis}{HTML}{4a4d9e}
\definecolor{Accent}{HTML}{0f0f45}
\definecolor{Body}{HTML}{555555}
\definecolor{Heading}{HTML}{181741}
\colorlet{heading}{Heading}
\colorlet{accent}{Accent}
\colorlet{emphasis}{Emphasis}
\colorlet{body}{Body}

% Change the bullets for itemize and rating marker
% for \cvskill if you want to
\renewcommand{\itemmarker}{{\small\textbullet}}
\renewcommand{\ratingmarker}{\faCircle}

%% sample.bib contains your publications
\addbibresource{sample.bib}

\usepackage[colorlinks]{hyperref}
\hypersetup{
	colorlinks=true,    
	urlcolor=blue,
}

\begin{document}

\name{Roberto Arias Ruiz}
\tagline{Software Product Manager}
\photo{2.5cm}{Rober_Gray_Round}
\personalinfo{%
  % Not all of these are required!
  % You can add your own with \printinfo{symbol}{detail}
  \mailaddress{roberlamerma@gmail.com}
  % \phone{+41 774967886 / +34 678714822}
  \phone{+41 774967886}
%  \mailaddress{Sonnhaldenstrasse 25}
%  \location{Sonnhaldenstrasse 25, 6331 Hünenberg, Switzerland}
  \linkedin{linkedin.com/in/roberlamerma}
  % \github{github.com/roberlamerma}
  %% You MUST add the academicons option to \documentclass, then compile with LuaLaTeX or XeLaTeX, if you want to use \orcid or other academicons commands.
%   \orcid{orcid.org/0000-0000-0000-0000}

	

}

%% Make the header extend all the way to the right, if you want. 
\begin{fullwidth}
\makecvheader
\end{fullwidth}

%% Provide the file name containing the sidebar contents as an optional parameter to \cvsection.
%% You can always just use \marginpar{...} if you do
%% not need to align the top of the contents to any
%% \cvsection title in the "main" bar.
\cvsection[page1sidebar]{Profile}
\begin{itemize} 
	\item Over 20 years of comprehensive experience in all stages of the Software Development Life Cycle: from conception to delivery.
	\item Ability to bridge the gap between technical teams and business stakeholders, ensuring that projects meet both strategic goals and user needs.
	\item Expert in Agile methodologies (SAFe, Scrum) with a proven track record of successful implementations.	       
	\item Exceptional communication and presentation skills, with the ability to engage both technical and non-technical audiences.
	\item Skilled in managing high-pressure situations with data and fact-driven decision-making.
\end{itemize}

\medskip

\cvsection{Experience}

%\cvevent{Product Owner}{Roche Diagnostics}{May 2011 -- Ongoing}{Barcelona, Spain}
%\cvevent{Roche Diagnostics \href{https://www.roche.com/about/business/diagnostics.htm}{\small <http://www.roche.com>}\par}{Product Owner}{May 2015 -- Ongoing}{Barcelona, Spain}
\cvevent{Roche Diagnostics}{Product Manager}{Nov 2022 -- Now}{Rotkreuz, Switzerland}
\begin{itemize}
	\item SW Product Manager for \textbf{Roche's first DNA-Sequencing instrument}.
	\item Developed project plans, including the product roadmap, work breakdown structure, and overall software vision. 
	\item Worked together with various stakeholders in order to define the right features for each of our milestones.
	\item Recruited and formed the SW development teams.
\end{itemize}


\cvevent{}{Product Owner}{May 2015 -- Nov 2022}{Rotkreuz, Switzerland}
\begin{itemize}
	\item Led product development for the {\href{https://www.cobasliat.com/}{\textsuperscript{\textregistered}Cobas \textsuperscript{\textregistered}Liat}} point-of-care device, which was \textbf{pivotal during the COVID-19 pandemic}.
	\item Defined and prioritized user stories for the SW development teams and coordinated acceptance testing with QA teams in a distributed setup (Spain and Switzerland).
	\item Created detailed requirements and user interface sketches to guide development.
\end{itemize}

\cvevent{}{Scrum \& Build Master / Senior Software Developer}{May 2011 -- May 2015}{Barcelona, Spain}
\begin{itemize}
	\item Led a local Scrum team and coordinated with international teams in Spain, India, and Switzerland to deliver a high-throughput PCR analyzer.
	\item Developed a \textbf{comprehensive reporting and KPI analysis suite} using Microsoft Reporting/Integration Services, TFS, SQL, OLAP cubes, and PowerBI.
\end{itemize}

\divider

\cvevent{Wolters Kluwer}{Software Architect}{Apr 2010 -- May 2011}{Barcelona, Spain}
\begin{itemize}
	\item Designed and developed a \textbf{C{\#}-based SW framework} for internal tax and accounting applications, enhancing operational efficiency.
\end{itemize}

\vfil \break

% \divider
\cvsection[page2sidebar]{Experience}

\cvevent{Amper}{Senior Software Developer}{Mar 2007 -- Feb 2010}{Barcelona, Spain}
\begin{itemize}
	\item Conducted \textbf{research and development} on embedded systems and comm protocols, driving innovation in the company’s product offerings.
	\item Designed and developed .NET services and applications.
\end{itemize}

\divider

\cvevent{Grupo 3C}{Software Developer}{Dec 2005 -- Jan 2007}{Barcelona, Spain}
\begin{itemize}
	\item Developed \textbf{web applications} using PHP, Perl, MySQL, HTML, CSS, and JavaScript, improving client service delivery.
\end{itemize}

\divider

\cvevent{iChameleon Group INC}{Full-stack Software Developer}{Aug 2004 -- Oct 2005}{(Freelancer - several locations)}
\begin{itemize}
	\item Designed and implemented SW for an \textbf{online music shop} as a full-stack developer, utilizing Perl, FastCGI, PostgreSQL, and Objective-C.
\end{itemize}

\divider

\cvevent{Gaiax Co, LTD}{Software Developer Internship}{Aug 2003 -- Jul 2004}{Tokyo, Japan}
\begin{itemize}
	\item Developed a Web game using Perl (mod perl and FastCGI). This was also \textbf{my university's thesis research project} on high loading \& demand technologies.
\end{itemize}

\divider

\cvevent{Computing Laboratory, Sim\'{o}n Bol\'{i}var University}{IT \& Systems Administrator}{Oct 1999 -- Oct 2002}{Caracas, Venezuela}
\begin{itemize}
	\item Managed and maintained parts of the \textbf{university’s LAN and lab servers} across Linux, Solaris, and Windows platforms, ensuring operational continuity and reliability.
	\item Coordinated the admission of new members (exams, training, mentoring)
\end{itemize}

\medskip
\medskip
%\medskip
%\medskip

% Adapted from @Jake's answer from http://tex.stackexchange.com/a/82729/226
% \wheelchart{outer radius}{inner radius}{
% comma-separated list of value/text width/color/detail}
\cvsection{Product management skills}
\wheelchart{1.5cm}{0.5cm}{%
	3/8em/Emphasis!90/Stakeholder management, 
	2/8em/Emphasis!45/KPI analysis,
	3/8em/Emphasis/Communication,
	3/4em/Emphasis!75/Usability \& UX,
	2/5em/Emphasis!30/\mbox{Requirements engineering},
%	2/10em/Emphasis!15/People leader,
	3/8em/Emphasis!60/Story mapping \& Roadmaps
}

\cvsection{Agile competencies}
\wheelchart{1.5cm}{0.5cm}{%
	3/8em/Emphasis!90/Lean-Agile Leadership, 
	2/8em/Emphasis!45/Expertise forming and managing Agile teams,
	3/8em/Emphasis/DevOps and Release on Demand,
	3/4em/Emphasis!75/\mbox{Business Solutions},
	2/5em/Emphasis!30/\mbox{Lean Systems Engineering},
	2/10em/Emphasis!15/People leader
}

%\cvsection{Programming skills}
% \wheelchart{1.5cm}{0.5cm}{%
% 	7/2em/Emphasis!80/C\#, 
% 	5/2em/Emphasis!50/Perl,
% 	3/2em/Emphasis!40/C++,
% 	3/2em/Emphasis/Bash,
% 	2/2em/Emphasis!60/Phyton,
% 	4/2em/Emphasis!20/{HTML/CSS/JS}
% }

% \cvsection{OS skills}

% \wheelchart{1.5cm}{0.5cm}{%
% 	10/2em/Emphasis!30/Linux, 
% 	3/2em/Emphasis!20/Mac OS,
% 	8/2em/Emphasis!60/Windows,
% 	5/2em/Emphasis/Embedded Linux
% }

%\colorlet{heading}{Heading}
%\colorlet{accent}{Accent}
%\colorlet{emphasis}{Emphasis}
%\colorlet{body}{Body}

%\clearpage
%\cvsection[page2sidebar]{Publications}

%\nocite{*}

%\printbibliography[heading=pubtype,title={\printinfo{\faBook}{Books}},type=book]

%\divider

%\printbibliography[heading=pubtype,title={\printinfo{\faFileTextO}{Journal Articles}},type=article]

%\divider

%\printbibliography[heading=pubtype,title={\printinfo{\faGroup}{Conference Proceedings}},type=inproceedings]

%% If the NEXT page doesn't start with a \cvsection but you'd
%% still like to add a sidebar, then use this command on THIS
%% page to add it. The optional argument lets you pull up the 
%% sidebar a bit so that it looks aligned with the top of the
%% main column.
% \addnextpagesidebar[-1ex]{page3sidebar}

\end{document}
