%%%%%%%%%%%%%%%%%
% This is an sample CV template created using altacv.cls
% (v1.1.3, 30 April 2017) written by LianTze Lim (liantze@gmail.com). Now compiles with pdfLaTeX, XeLaTeX and LuaLaTeX.
% 
%% It may be distributed and/or modified under the
%% conditions of the LaTeX Project Public License, either version 1.3
%% of this license or (at your option) any later version.
%% The latest version of this license is in
%%    http://www.latex-project.org/lppl.txt
%% and version 1.3 or later is part of all distributions of LaTeX
%% version 2003/12/01 or later.
%%%%%%%%%%%%%%%%

%% If you need to pass whatever options to xcolor
\PassOptionsToPackage{dvipsnames}{xcolor}

%% If you are using \orcid or academicons
%% icons, make sure you have the academicons 
%% option here, and compile with XeLaTeX
%% or LuaLaTeX.
% \documentclass[10pt,a4paper,academicons]{altacv}

%% Use the "normalphoto" option if you want a normal photo instead of cropped to a circle
% \documentclass[10pt,a4paper,normalphoto]{altacv}

\documentclass[10pt,a4paper]{altacv}
%% AltaCV uses the fontawesome and academicon fonts
%% and packages. 
%% See texdoc.net/pkg/fontawecome and http://texdoc.net/pkg/academicons for full list of symbols.
%% 
%% Compile with LuaLaTeX for best results. If you
%% want to use XeLaTeX, you may need to install
%% Academicons.ttf in your operating system's font 
%% folder.


% Change the page layout if you need to
\geometry{left=1cm,right=9cm,marginparwidth=6.8cm,marginparsep=1.2cm,top=1.25cm,bottom=1.25cm,footskip=2\baselineskip}

% Change the font if you want to.

% If using pdflatex:
\usepackage[T1]{fontenc}
\usepackage[utf8]{inputenc}
\usepackage[default]{lato}

% For justifying
\usepackage{ragged2e}

% If using xelatex or lualatex:
% \setmainfont{Lato}

% Change the colours if you want to
\definecolor{Emphasis}{HTML}{4a4d9e}
\definecolor{Accent}{HTML}{0f0f45}
\definecolor{Body}{HTML}{666666}
\definecolor{Heading}{HTML}{181741}
\colorlet{heading}{Heading}
\colorlet{accent}{Accent}
\colorlet{emphasis}{Emphasis}
\colorlet{body}{Body}

% Change the bullets for itemize and rating marker
% for \cvskill if you want to
\renewcommand{\itemmarker}{{\small\textbullet}}
\renewcommand{\ratingmarker}{\faCircle}
%% sample.bib contains your publications
\addbibresource{sample.bib}

\usepackage[colorlinks]{hyperref}
\hypersetup{
	colorlinks=true,    
	urlcolor=blue,
}

\begin{document}

\name{Roberto Arias Ruiz}
\tagline{Software Engineer - Product Owner}
\photo{2.8cm}{RoberAzul_Round}
\personalinfo{%
  % Not all of these are required!
  % You can add your own with \printinfo{symbol}{detail}
  \mailaddress{roberlamerma@gmail.com}
  \phone{+34 678714822}
%  \mailaddress{Sonnhaldenstrasse 25}
  \location{Sonnhaldenstrasse 25, 6331 Hünenberg, Switzerland}
  \twitter{@roberlamerma}
  \linkedin{linkedin.com/in/roberlamerma}
  \github{github.com/roberlamerma}
  %% You MUST add the academicons option to \documentclass, then compile with LuaLaTeX or XeLaTeX, if you want to use \orcid or other academicons commands.
%   \orcid{orcid.org/0000-0000-0000-0000}
}

%% Make the header extend all the way to the right, if you want. 
\begin{fullwidth}
\makecvheader
\end{fullwidth}

%% Provide the file name containing the sidebar contents as an optional parameter to \cvsection.
%% You can always just use \marginpar{...} if you do
%% not need to align the top of the contents to any
%% \cvsection title in the "main" bar.
\cvsection[page1sidebar]{Profile}

\begin{itemize} 
	\item Broad experience in all Software Development Life-cycle milestones.
	\item Expert on Scrum \& Agile methodologies.
	\item Ability to cope with critical and high pressure situations.              
	\item Excellent communication and presentation skills.
	\item Data \& fact driven. Interested in Data Science and Machine Learning.
\end{itemize}

\medskip

\cvsection{Experience}

%\cvevent{Product Owner}{Roche Diagnostics}{May 2011 -- Ongoing}{Barcelona, Spain}
%\cvevent{Roche Diagnostics \href{https://www.roche.com/about/business/diagnostics.htm}{\small <http://www.roche.com>}\par}{Product Owner}{May 2015 -- Ongoing}{Barcelona, Spain}
\cvevent{Roche Diagnostics}{Product Owner}{May 2015 -- Ongoing}{Rotkreuz, Switzerland}
\begin{itemize}
	\item Product owner for a Point of Care device that automates virus and disease detection using molecular (PCR) analysis. Stakeholder management.
	\item Work Packages/User Stories creation and later acceptance for 3 teams on a distributed location setup (Spain and Switzerland)
	\item Requirements and usability (UX) engineering: requirements management \& creation of User Interface Sketches.
	\item Creation of Roadmap plans and product vision, using proven Project Management artifacts. Resource \& budget management.
\end{itemize}

\cvevent{}{Scrum \& Build Master / Senior Software Developer}{May 2011 -- May 2015}{Barcelona, Spain}
\begin{itemize}
	\item Coordination of a local 9 people Scrum team with several teams, spread along Spain, India and Switzerland, for a high-throughput diagnostics analyzer.
	\item Project management reports and KPI analysis, using Microsoft Reporting/Integration Services, TFS (SQL and OLAP cubes), Tableau and Excel.
\end{itemize}

\divider

\cvevent{Wolters Kluwer}{Software Architect}{Apr 2010 -- May 2011}{Barcelona, Spain}
\begin{itemize}
	\item Analysis and Design for a core development software framework in a joint project among all the European branches of the company.
	\item Technologies: .Net, C\#, TFS, SQL Server, WCF, WPF, EntityFramework.
\end{itemize}


\divider

\cvevent{Amper}{Senior Software Developer}{Mar 2007 -- Feb 2010}{Barcelona, Spain}
\begin{itemize}
	\item Design of .NET services and applications: client-server solutions, MVC, distributed applications, high availability. 
	\item Research and Development of embedded systems and mobile applications.
\end{itemize}

\divider

\cvevent{Grupo 3C}{Software Developer}{Dec 2005 -- Jan 2007}{Barcelona, Spain}
\begin{itemize}
	\item Web: PHP, Perl, MySQL, HTML, CSS, JS. Web traffic management (Apache)
\end{itemize}

\cvsection[page2sidebar]{Experience}

%\divider

\cvevent{iChameleon Group INC}{Freelance Software Developer}{Aug 2004 -- Oct 2005}{(Several locations)}
\begin{itemize}
	\item Freelance project for the development (Frontend and Backend) of an online music shop. Perl, FastCGI, PostgreSQL.
	\item Development of Mac applications (Objective-C, Cocoa Framework).
\end{itemize}

\divider

\cvevent{Gaiax Co, LTD}{Software Developer Internship}{Aug 2003 -- Jul 2004}{Tokyo, Japan}
\begin{itemize}
	\item Web game development using Perl (mod perl and FastCGI)
	\item Research project on high processing capacity technologies. Benchmark and Profiling of Web applications.
\end{itemize}

\divider

\cvevent{Computing Laboratory, Sim\'{o}n Bol\'{i}var University}{Linux System Administrator}{Oct 1999 -- Oct 2002}{Caracas, Venezuela}
\begin{itemize}
	\item Design and maintenance of several nodes of the university LAN. Maintenance of all software (Linux, Solaris and Windows servers)
	\item Responsible for the admission of new members (exams, training, mentoring)
\end{itemize}

\medskip

% Adapted from @Jake's answer from http://tex.stackexchange.com/a/82729/226
% \wheelchart{outer radius}{inner radius}{
% comma-separated list of value/text width/color/detail}
\cvsection{Project management skills}

\wheelchart{1.5cm}{0.5cm}{%
	2/8em/Emphasis!90/Stakeholder management, 
	3/8em/Emphasis!45/KPI analysis,
	3/8em/Emphasis/Communication,
	2/4em/Emphasis!75/Usability \& UX,
	3/5em/Emphasis!30/\mbox{Requirements engineering},
	2/10em/Emphasis!15/People leader,
	3/8em/Emphasis!60/Story mapping \& Roadmaps
}

\cvsection{Programming skills}

\wheelchart{1.5cm}{0.5cm}{%
	7/2em/Emphasis!80/C\#, 
	5/2em/Emphasis!50/Perl,
	3/2em/Emphasis!40/C++,
	3/2em/Emphasis/Bash,
	2/2em/Emphasis!60/Phyton,
	4/2em/Emphasis!20/{HTML/CSS/JS}
}

\cvsection{OS skills}

\wheelchart{1.5cm}{0.5cm}{%
	10/2em/Emphasis!30/Linux, 
	3/2em/Emphasis!20/Mac OS,
	8/2em/Emphasis!60/Windows,
	5/2em/Emphasis/Embedded Linux
}

%\colorlet{heading}{Heading}
%\colorlet{accent}{Accent}
%\colorlet{emphasis}{Emphasis}
%\colorlet{body}{Body}

%\clearpage
%\cvsection[page2sidebar]{Publications}

%\nocite{*}

%\printbibliography[heading=pubtype,title={\printinfo{\faBook}{Books}},type=book]

%\divider

%\printbibliography[heading=pubtype,title={\printinfo{\faFileTextO}{Journal Articles}},type=article]

%\divider

%\printbibliography[heading=pubtype,title={\printinfo{\faGroup}{Conference Proceedings}},type=inproceedings]

%% If the NEXT page doesn't start with a \cvsection but you'd
%% still like to add a sidebar, then use this command on THIS
%% page to add it. The optional argument lets you pull up the 
%% sidebar a bit so that it looks aligned with the top of the
%% main column.
% \addnextpagesidebar[-1ex]{page3sidebar}

\end{document}
